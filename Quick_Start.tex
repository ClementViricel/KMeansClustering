\documentclass[12pt]{article}
%\usepackage[utf8]{inputenc}
\usepackage{verbatim}
\usepackage{amsmath}
\usepackage{amssymb}
\usepackage{amsfonts}
\usepackage{amsthm}
\usepackage{amsbsy}

\title{Quick Start}
\author{Cl\'ement Viricel}

\begin{document}
\maketitle
\section{Input File}
Let's take a look at the file which the program eat. It has the following format:
\begin{verbatim}
Nb_of_point Nb_of_dimension Nb_of_Clusters Nb_of_Iteration HasName
x1 x2 ... (Name)
x1 x2 ... (Name)
x1 x2 ... (Name)
.
.
.
\end{verbatim}
For example, if you are trying to run the program for $1000$ iteration over one hundred $\mathbb{R}^3$ coordinates in $4$ clusters the file will look like:
\begin{verbatim}
100 3 4 1000 0
x y z
x y z
.
.
.
\end{verbatim}
It's an easy and common way to represent the data but the parser can be changed if needed (with more work). \\
Adding a name to the point can allow user to see if there is a pattern with its data. For exemple, one can cluster 1000 measures of any dimension of $N$ different types of any kind and clusterize it into $N$ clusters to see if the measure are strongly correlated to the type.
\section{Output File}
The output file produce by the program is as simple as the input file. It will write in ``\textit{Results.txt}'' each cluster with its points and its central value (that correspond to te barycenter of the cluster, aka central value, aka representative value :
\begin{verbatim}
Cluster i
Point p : x1 x2 ... (Name)
.
.
Central point
\end{verbatim}
\section{The program}
To compile the program please enter the following command:
\begin{verbatim}
g++ src/*.cpp -o kmeans
\end{verbatim}
To run the program, it's really easy, just enter:
\begin{verbatim}
./kmeans [input_file]
\end{verbatim}
Enjoy !
\end{document}